\documentclass[a4,center,fleqn]{NAR}


\usepackage{NAR-natbib}
\bibliographystyle{unsrtnat}
\usepackage{color}
\newcommand{\rngcomment}[1]{{\color{red}RNG: #1}}
\newcommand{\bkmcomment}[1]{{\color{blue}BKM: #1}}
% Enter dates of publication
\copyrightyear{2008}
\pubdate{31 July 2009}
\pubyear{2009}
\jvolume{37}
\jissue{12}

\articlesubtype{Genomics and Bioinformatics}

\begin{document}

\title{Article title}

\author{%
Brian K. Mannakee\,$^{1}$ and
Ryan N. Gutenkunst\,$^{2}$%
\footnote{To whom correspondence should be addressed.
Email: rgutenk@email.arizona.edu}}

\address{%
$^{1}$University of Arizona Mel and Enid Zuckerman College of Public Health
and
$^{2}$University of Arizona Department of Molecular and Cellular Biology}
% Affiliation must include:
% Department name, institution name, full road and district address,
% state, Zip or postal code, country

\history{%
Received January 1, 2009;
Revised February 1, 2009;
Accepted March 1, 2009}

\maketitle

\begin{abstract}
Text. Text. Text. Text. Text. Text. Text. Text. Text. Text. Text.
Text. Text. Text. Text. Text. Text. Text. Text. Text. Text. Text.
Text. Text. Text. Text. Text. Text. Text. Text. Text. Text. Text.
Text. Text. Text. Text. Text. Text. Text. Text. Text. Text. Text.
Text. Text. Text. Text. Text. Text. Text. Text. Text. Text. Text.
Text. Text. Text. Text. Text. Text. Text. Text. Text. Text. Text.
Text. Text. Text. Text. Text. Text. Text. Text. Text. Text. Text.
Text. Text. Text. Text. Text. Text. Text. Text. Text. Text. Text.
Text. Text.
\end{abstract}


\section{Introduction}

Cancer develops as the result of the accumulation of somatic mutations and clonal selection of cells with mutations that confer a selective advantage on the cell.
Understanding the forces that shaped the evolutionary history of a tumor, the mutations that are responsible for its growth, the rate at which mutations are occurring, or how much genetic diversity is likely present in the tumor, requires accurate variant calling, particularly at low variant allele frequency \cite{Williams2016,Bozic2016,Williams2018}.
Accurate variant identification is also critical in optimizing the treatment regime for an individual patients disease \citep{Ding2012,Mardis2012,Chen2013,Borad2014,Findlay2016}.
Low frequency mutations present a significant problem for current mutation calling methods because their signature in the data is difficult to distinguish from the noise introduced by Next Generation Sequencing (NGS), and this problem increases as sequencing depth increases.

Methods for identifying true somatic mutations - i.e. variant calling -  from NGS data are an active area of research in bioinformatics.
The earliest widely used somatic variant callers aimed specifically at tumors, Mutect1 and Varscan2, used a combination of heuristic filtering and a model of sequencing errors to identify and score potential variants, setting a threshold for that score designed to balance sensitivity and specificity \citep{Koboldt2012,Cibulskis2013}.
Subsequent research gave rise to a number of alternate variant calling strategies including haplotype based callers \citep{Garrison2012},
joint genotype analysis (SomaticSniper, JointSNVMix2, Seurat, and CaVEMan,MuClone)\citep{Larson2012,Roth2012a,Christoforides2013,Jones2016,Dorri2019}, allele frequency based analysis (Strelka, MuTect, LoFreq, EBCall, deepSNV, LoLoPicker, and MuSE)\citep{Saunders2012,Wilm2012,Shiraishi2013b,Gerstung2012,Carrot-Zhang2017,Fan2016}, and a mixture of ensemble and deep learning methods (MutationSeq, SomaticSeq, SNooPer, and BAYSIC).
All of these methods have varying levels of complexity, and some are focused on specific types of data.
The one thing they all have in common is that they either implicitly or explicitly assume that the probability of a mutation occuring at a particular site is proportional to the overall mutation rate, and the same at every site in the genome.

Single nucleotide substitions, i.e. simple mutations, arise in tumors at a rate and at genomic locations driven by two main processes. 
The first is the spontaneous accumulation of mutations that occurs in all dividing tissues, and has a characteristic mutation signature that describes the probability of mutation in a given genomic context \citep{Nik-Zainal2012a,Alexandrov2015,Lee-Six2018}. 
The second, and far more complex, process is the accumulation of mutations through exposure to mutagens or degradation - via mutation or deletion - of cellular machinery responsible for the identification and repair of damage or replication errors. 
Many mutagens and DNA repair mechanism defects also have highly specific mutation signatures, such that they can be identified by observing the mutations in the tumor \citep{Alexandrov2013a,Helleday2014a,Nik-Zainal2016,Kandoth2013,Alexandrov2016}.

Here we present an algorithm for estimating the prior probability of mutation at a given site using the observed mutation spectrum of the tumor as well as its mutation rate, and show that the addition of this prior to the MuTect variant calling model produces a superior variant classifier in both simulated and real tumor data.
We then extend the method with an application of the local false discovery rate by computing the probability that a site is non-null under an assumption of clonal expansion with either early or small selective differences between clones.
We provide a simple implementation in R that takes MuTect caller output as input, and returns the posterior probability that a site is variant for every site observed by MuTect.
\newpage
\section{MATERIALS AND METHODS}
\subsection{Somatic variant calling base probability model}

\bkmcomment{I feel like a discussion of the mutation signature should go first. Referencing the figure? The issue is that I have this m,M notation but the context C rolls in the m.}
At every site in the genome with non-zero coverage, Next Generation Sequencing (NGS) produces a vector $\mathbf{x}  = (\{b_i\},\{q_i\}), i = 1\dots D$ of base calls and their associated quality scores, where $D$ is total read depth.
The goal is to use $\mathbf{x}$ to select between competing hypotheses;
$$
  \begin{array}{l}
    \mathbf{H_0}:\quad \textrm{Alt allele} = m;\quad\nu = 0\\
    \mathbf{H_1}:\quad \textrm{Alt allele} = m;\quad\nu = \hat{f},
  \end{array}
$$
where $\nu$ is the variant allele frequency, $\hat{f}$ is the maximum likelihood estimate of $\nu$ given data $\mathbf{x}$, i.e. the ratio of the count of variant reads and total read depth, and $m$ is any of the 3 possible alternate non-reference bases.
For a given read with base $b_i$ and q-score $q_i$, the density function under a particular hypothesis is defined as

Making the common assumption that NGS reads are independent, the posterior probability of a given hypothesis is 
$$
  \begin{array}{rl}
    P(m,\nu) = &p(m,\nu) \cdot \mathcal{L}_{m,\nu}(\mathbf{x}) \\ \\
             = &p(m,\nu) \cdot \prod_{i=1}^{D} \textrm{f}_{m,\nu}(x_i),
  \end{array}
$$
where the density $\textrm{f}_{m,\nu}(x_i)$ is

$$
  \textrm{f}_{m,\nu}({b_i,q_i}) = \left\{
    \begin{array}{cr}
      \nu \frac{10^{-q_i/10}}{3} + (1-\nu)(1-10^{-q_i/10}) & b_i = \textrm{reference}\\ \\
      \nu(1-10^{-q_i/10}) + (1-\nu) \frac{10^{-q_i/10}}{3} & b_i = m\\ \\
      \frac{10^{-q_i/10}}{3} & otherwise.
    \end{array}
    \right.
$$

Assuming that the identity of the alternate allele and the estimated allele frequency are independent of each other, and $\nu$ is uniformly distributed results in the final model
$$
  P(m,\nu) = p(m) \cdot \mathcal{L}_{m,\nu}(\mathbf{x}).
$$

\subsection{Site-specific prior probability of mutation}

The probability which we have denoted $p(m)$ for compactness above is actually the joint probability that a mutation has occured $M$, and that it was to allele $m$, which we will denote here $p(m,M)$.
However, $p(m,M)$ is not constant for every site in the genome, and follows a distribution -- conditional on the genomic context surrounding the site -- described by the mutation signature of the tumor and denoted $p(m,M | C)$(citations).
The prior probability $\mathrm{p}(m,M \mid C)$ can be decomposed as
$$
\mathrm{p}(m,M \mid C) = \mathrm{p}(C \mid M,m) \frac{p(m,M)}{p(C)}
$$
\bkmcomment{This is a problem. I am having trouble with the fact that C includes m. Maybe $p(C | M)$}

\subsection{Estimation of the mutation rate.}

As discussed above, MuTect fixes the site-specific mutation probability at $\mathrm{p}(M)= \mu = 3\mathrm{e}{-6}$.
All variant callers we are aware of either fix this parameter $\mu$ or allow the user to input the value, but there is no way to really know this value until the variant allele frequencies have been observed.
As with estimation of the mutation signature, we can use high confidence mutations and a model of tumor evolution to compute the tumor-specific mutation rate.
\citet{Bozic2016} show that for any variant allele frequency $\alpha$, the total number of mutations with frequency greater than $\alpha$ and less than $0.25$ is

$$
N(\alpha) = N\mu \left( \frac{1}{\alpha} - \frac{1}{0.25} \right)
$$

Where $N$ is the total number of sites sequenced and $\mu$ is the per-site mutation probability.
By selecting $\alpha$ such that we are highly confident in all calls at frequencies greater than $\alpha$, we can compute $\mu$ and recompute the odds in favor of $\mathbf{H_1}$.


%`\subsection{False positive rate control.}

%We develop a method, following \citet{Efron2008}, for controlling the false positive rate.
%Every site with sufficient coverage and at least 1 alternate read falls into one of two classes, they are either \textit{null} (non-variant with $\nu = 0$) or \textit{nonnull} (variant with $\nu = \hat{f}$) with prior probabilities $p_0$ and $p_1 = 1-p_0$,

%$$
%\begin{array}{ll}
%p_0 = \textrm{P}\{\textrm{null}\} \quad & \textrm{f}_{0}(\mathbf{x}) \quad \textrm{density if null}\\
%p_1 = \textrm{P}\{\textrm{nonnull}\} \quad & \textrm{f}_{1}(\mathbf{x}) \quad \textrm{density if nonnull} .
%\end{array}
%$$

%The local, or site-specific, true positive probability $p_1$ can be estimated as the fraction of all sequenced sites that are expected to be positive.
%In a neutrally evolving tumor, the number of cells is growing exponentially, and the count of variants with an allele frequency greater than a given allele frequency $f$ is \cite{Bozic2016,Williams2016a}.
%$$
%N(f) = \frac{N\mu}{f},
%$$
%Where $N$ is the total number of sites sequenced and $\mu$ is the per-site mutation probability.
%The estimated fraction of all of the sites in the genome that will have a mutation with frequency $f$ is
%$$
%\hat{p}_1 = \frac{\int_{f_{-}}^{f_{+}} N(f)}{N} = \frac{\mu}{f - .01} -  \frac{\mu}{f + .01}
%$$

%\bkmcomment{This is wrong. This is the cdf. I need the integral of the derivative which is the density.}
%and the estimated null probability $\hat{p}_0 = 1 - \hat{p}_1$.
%\citet{Williams2016a} provides a full derivation for this, we are essentially computing the integral here of $N(f)$ in a small area around $f$.


\subsection{Tumor simulations.}

We simulated realistic variant sites and allele frequencies using a branching process to simulate neutral evolution with no death.
Variants were selected from TCGA and PCAWG variant files(dates).
Whole genome (100X depth), and whole exome (500X depth) reads from the GRCH38 reference genome with VarSim \cite{Mu2015}, and aligned them to GRch38 with BWA \cite{Li2009a}, both with default parameters.
Variants were spiked to create tumors with Bamsurgeon with default parameters \cite{Ewing2015a},
and called with MuTect 1.1.7 \cite{Cibulskis2013} with the following parameters:

\begin{tiny}
\begin{verbatim}

  java -Xmx24g -jar $MUTECT_JAR --analysis_type MuTect --reference_sequence $ref_path \
        --dbsnp $db_snp \
        --enable_extended_output \
        --fraction_contamination 0.00 \
        --tumor_f_pretest 0.00 \
        --initial_tumor_lod -10.00 \
        --required_maximum_alt_allele_mapping_quality_score 1 \
        --input_file:normal $tmp_normal \
        --input_file:tumor $tmp_tumor \
        --out $out_path/$chr.txt \
        --coverage_file $out_path/$chr.cov

\end{verbatim}
\end{tiny}
Variants identified by MuTect are labelled as to whether they pass all MuTect filters, pass all filters *other* than the evidence threshold \textrm{tlod\_f\_star}, or fail to pass any filter other than \textrm{tlod\_f\_star}. Variants that pass all filters or fail only \textrm{tlod\_f\_star} are then passed to {method} for prior estimation and rescoring.

\subsection{Real tumor data.}
\subsubsection{Acute Myeloid Leukemia}
We downloaded the whole genome sequence for aml31 \bkmcomment{citation and download date, and primary/relapse distinction} \cite{Griffith2015}.
We merged the gold and platinum lists, and define as \textit{not present} any variant for which deep sequencing was performed and zero alternate reads were observed.
This is a very conservative metric.
\subsubsection{What I call the cell paper} Not sure if this will get used.

\subsection{Calibration}
We use the Integrated Calibration Index to quantify the difference between MuTect and the Calibrated model \cite{Austin2019}.
Briefly, a loess-smoothed regression is fit by regressing the binary (T/F) variant classification against the reported posterior probability of a true variant for both MuTect and the Calibrated model (Figure \ref{NAR-fig2}\textbf{b,d}).
Plotting the smoothed regression predictions against the actual predictions would result in a diagonal line, and the Integrated Calibration Index is the total area between the result and a diagonal line.

\begin{figure}[t]
\begin{center}
\includegraphics{figures/signature_figure.pdf}
\end{center}
\caption{Signature words.}
\label{NAR-fig1}
\end{figure}


\section{RESULTS}
\bkmcomment{I can't find a measured mutation rate for aml31. I get 3e-8 which isn't out of line for a leukemia primary.}

\subsection{Convergence of the mutation prior to the true mutation profile}

The dirichlet prior probability of mutation converges quickly to the true mutation signature.
The rate and degree of convergence are correlated with the concentration of the mutation signature operating in the tumor (Figure \ref{NAR-fig1}\textbf{e,f}).
\subsection{Performance on real data}

\begin{itemize}
  \item Similar performance in aml31
  \item Similar calibration
  \item caveats about the fact that we don't have true false positive/false negative data
\end{itemize}



\subsection{Simulated data}

\subsubsection{classification}
We simulated both whole genome and whole exome tumors with three different mutation signatures.
Measures of classification performance are positively correlated with signature concentration.
For all simulations the calibrated model produces a better variant classifier than MuTect (Figure \ref{NAR-fig2} \textbf{a} and \textbf{c}).

\subsubsection{calibration}
We computed the Integrated Calibration Index \cite{Austin2019} for both simulated (Figure \ref{NAR-fig2} \textbf{b,d}) and real tumors \bkmcomment{aml31 when the plot is ready.}.
In all cases we find that the Calibrated model is better calibrated than MuTect.\bkmcomment{not perfect though}

\begin{table*}[b]
\tableparts{%
\caption{This is a table with auroc, auprc, ICI, and maybe fraction called?}
\label{table:01}%
}{%
\begin{tabular*}{\textwidth}{@{}lllllllll@{}}
\toprule
Depth & Signature & $\mu$ & AUROC &  & AUPRC & & ICI & 
\\
& & (estimated) & MuTect & Calibrated & MuTect & Calibrated & MuTect & Calibrated
\\
\colrule
100X WGS & Concentrated & Row 1 & Row 1 & -- & -- &&
\\
100X WGS & Intermediate & Row 1 & Row 1 & -- & -- &&
\\
100X WGS & Diffuse & Row 1 & Row 1 & -- & -- &&
\\
500X WES & Concentrated & Row 1 & Row 1 & -- & -- &&
\\
500X WES & Intermediate & Row 1 & Row 1 & -- & -- &&
\\
500X WES & Diffuse & Row 1 & Row 1 & -- & -- &&
\\
AML 31 & Actual & $3.6e-8$ & Row 1 & -- & -- &&
\\
\botrule
\end{tabular*}%
}
{$\mu$ = per-base mutation rate, AUROC/AUPRC = area under ROC/PRC curve, ICI = Integrated calibration index.}
\end{table*}


\begin{figure*}[t]
\begin{center}
\includegraphics{figures/fig2.pdf}
\end{center}
\caption{Model performance on 500X simulated whole exome.
\textbf{(a)} Precision recall curves and \textbf{(c)} Reciever operating characteristic curves for 3 mutation signatures.
Distribution of estimated true positive probabilities for true positive (top) and true negative (bottom) variants for \textbf{b)} the MuTect model and \textbf{d)} the Calibrated model.
A perfectly calibrated model would generate the diagonal line.
}
\label{NAR-fig2}
\end{figure*}


\section{DISCUSSION}
\begin{figure}[t]
  \begin{center}
  \includegraphics{figures/aml_plot.pdf}
  \end{center}
  \caption{aml result words.}
  \label{NAR-fig3}
  \end{figure}
\subsection{Discussion subsection one}

Text. Text. Text. Text. Text. Text. Text. Text. Text. Text. Text.
Text. Text. Text. Text. Text. Text. Text. Text. Text. Text. Text.
Text. Text. Text. Text. Text. Text. Text. Text. Text. Text. Text.
Text. Text. Text. Text. Text. Text. Text. Text. Text. Text. Text.
Text. Text. Text. Text. Text. Text. Text. Text. Text. Text. Text.
Text. Text. Text. Text. Text. Text. Text. Text. Text. Text. Text.
Text. Text. Text. Text. Text. Text. Text. Text. Text. Text. Text.
Text. Text. Text. Text. Text. Text. Text. Text. Text. Text. Text.
Text. Text. Text. Text. Text. Text. Text. Text. Text. Text. Text.
Text. Text. Text. Text. Text. Text. Text. Text. Text. Text. Text.
Text. Text. Text. Text. Text. Text. Text. Text. Text. Text. Text.
Text. Text. Text. Text. Text. Text. Text. Text. Text. Text. Text.
Text. Text. Text. Text. Text. Text. Text. Text. Text. Text. Text.
Text. Text. Text. Text. Text. Text. Text. Text. Text. Text. Text.
Text. Text. Text. Text. Text. Text. Text. Text. Text. Text. Text.
Text. Text. Text. Text. Text. Text. Text. Text. Text. Text. Text.
Text. Text. Text. Text. Text. Text. Text. Text. Text. Text. Text.
Text. Text. Text. Text. Text. Text. Text. Text. Text. Text. Text.
Text. Text. Text. Text. Text. Text. Text. Text. Text. Text. Text.
Text. Text. Text. Text. Text. Text. Text. Text. Text. Text. Text.
Text. Text. Text. Text. Text. Text.


\subsection{Discussion subsection two}

Text. Text. Text. Text. Text. Text. Text. Text. Text. Text. Text.
Text. Text. Text. Text. Text. Text. Text. Text. Text. Text. Text.
Text. Text. Text. Text. Text. Text. Text. Text. Text. Text. Text.
Text. Text. Text. Text. Text. Text. Text. Text. Text. Text. Text.
Text. Text. Text. Text. Text. Text. Text. Text. Text. Text. Text.
Text. Text. Text. Text. Text. Text. Text. Text. Text. Text. Text.
Text. Text. Text. Text. Text. Text. Text. Text. Text. Text. Text.
Text. Text. Text. Text. Text. Text. Text. Text. Text. Text. Text.
Text. Text. Text. Text. Text. Text. Text. Text. Text. Text. Text.
Text. Text. Text. Text. Text. Text. Text. Text. Text. Text. Text.
Text.

Text. Text. Text. Text. Text. Text. Text. Text. Text. Text. Text.
Text. Text. Text. Text. Text. Text. Text. Text. Text. Text. Text.
Text. Text. Text. Text. Text. Text. Text. Text. Text. Text. Text.
Text. Text. Text. Text. Text. Text. Text. Text. Text. Text. Text.
Text. Text. Text. Text. Text. Text. Text. Text. Text. Text. Text.
Text. Text. Text. Text. Text. Text. Text. Text. Text. Text. Text.
Text. Text. Text. Text. Text. Text. Text. Text. Text. Text. Text.
Text. Text. Text. Text. Text. Text. Text. Text. Text. Text. Text.
Text. Text. Text. Text. Text. Text. Text. Text. Text. Text. Text.
Text. Text. Text. Text. Text. Text. Text. Text. Text. Text. Text.
Text. Text. Text. Text. Text. Text. Text. Text. Text. Text.


\subsection{Discussion subsection three}

Text. Text. Text. Text. Text. Text. Text. Text. Text. Text. Text.
Text. Text. Text. Text. Text. Text. Text. Text. Text. Text. Text.
Text. Text. Text. Text. Text. Text. Text. Text. Text. Text. Text.
Text. Text. Text. Text. Text. Text. Text. Text. Text. Text. Text.
Text. Text. Text. Text. Text. Text. Text. Text. Text. Text. Text.
Text. Text. Text. Text. Text. Text. Text. Text. Text. Text. Text.
Text. Text. Text. Text. Text. Text. Text. Text. Text. Text. Text.
Text. Text. Text. Text. Text. Text. Text. Text. Text. Text. Text.
Text. Text. Text. Text. Text. Text. Text. Text. Text. Text. Text.
Text. Text. Text. Text. Text. Text. Text. Text. Text. Text. Text.
Text. Text. Text. Text. Text. Text. Text. Text. Text.

Text. Text. Text. Text. Text. Text. Text. Text. Text. Text. Text.
Text. Text. Text. Text. Text. Text. Text. Text. Text. Text. Text.
Text. Text. Text. Text. Text. Text. Text. Text. Text. Text. Text.
Text. Text. Text. Text. Text. Text. Text. Text. Text. Text. Text.
Text. Text. Text. Text. Text. Text. Text. Text. Text. Text. Text.
Text. Text. Text. Text. Text. Text. Text. Text. Text. Text. Text.
Text. Text. Text. Text. Text. Text. Text. Text. Text. Text. Text.
Text. Text. Text. Text. Text. Text. Text.

Text. Text. Text. Text. Text. Text. Text. Text. Text. Text. Text.
Text. Text. Text. Text. Text. Text. Text. Text. Text. Text. Text.
Text. Text. Text. Text. Text. Text. Text. Text. Text. Text. Text.
Text. Text. Text. Text. Text. Text. Text. Text. Text. Text. Text.
Text. Text. Text. Text. Text. Text. Text. Text. Text. Text. Text.
Text. Text. Text. Text. Text. Text. Text. Text. Text. Text. Text.
Text. Text. Text. Text. Text. Text. Text. Text. Text. Text. Text.
Text. Text. Text. Text. Text. Text. Text.


\section{CONCLUSION}

Text. Text. Text. Text. Text. Text. Text. Text. Text. Text. Text.
Text. Text. Text. Text. Text. Text. Text. Text. Text. Text. Text.
Text. Text. Text. Text. Text. Text. Text. Text. Text. Text. Text.
Text. Text. Text. Text. Text. Text. Text. Text. Text. Text. Text.
Text. Text. Text. Text. Text. Text. Text. Text. Text. Text. Text.
Text. Text. Text. Text. Text. Text. Text. Text. Text. Text. Text.
Text. Text. Text. Text. Text. Text. Text. Text. Text. Text. Text.
Text. Text. Text. Text. Text. Text. Text. Text. Text. Text. Text.
Text. Text. Text. Text. Text. Text. Text. Text. Text. Text. Text.
Text. Text. Text.


\section{ACKNOWLEDGEMENTS}

Text. Text. Text. Text. Text. Text. Text. Text. Text. Text. Text.
Text. Text. Text. Text.


\subsubsection{Conflict of interest statement.} None declared.




\bibliography{caller_paper}



\end{document}
